%% -*-LaTeX-*-
%%% newwaitingtimeabsorbtion.tex.orig
%%% Prettyprinted by texpretty lex version 0.02 [21-May-2001]
%%% on Tue Jun  6 05:30:35 2017
%%% for Steve Dunbar (sdunbar@family-desktop)

\documentclass[12pt]{article}

\input{../../../../etc/macros} %\input{../../../../etc/mzlatex_macros}
\input{../../../../etc/pdf_macros}

\bibliographystyle{plain}

\begin{document}

\myheader \mytitle

\hr

\sectiontitle{Waiting Time to Absorption}

\hr

\usefirefox

\hr

% \visual{Study Tip}{../../../../CommonInformation/Lessons/studytip.png}
% \section*{Study Tip}

% \hr

\visual{Rating}{../../../../CommonInformation/Lessons/rating.png}
\section*{Rating} %one of
% Everyone: contains no mathematics.
% Student: contains scenes of mild algebra or calculus that may require guidance.
Mathematically Mature:  may contain mathematics beyond calculus with
proofs.  % Mathematicians Only: prolonged scenes of intense rigor.

\hr

\visual{Section Starter Question}{../../../../CommonInformation/Lessons/question_mark.png}
\section*{Section Starter Question}

For a Markov chain with an absorbing state, describe the random variable
for the time until the chain gets absorbed.

\hr

\visual{Key Concepts}{../../../../CommonInformation/Lessons/keyconcepts.png}
\section*{Key Concepts}

\begin{enumerate}
    \item
        Let \( \{ X_n \} \) be a finite-state absorbing Markov chain with \( a \)
        absorbing states and \( t \) transient states.  Let the \( (a + t)
        \times (a + t) \) transition probability matrix be \( P \).  Order the
        states so the absorbing states come first and non-absorbing, i.e.\
        transient, states come last. 
        Then the transition probability matrix has the block-matrix form
        \[
            P =
            \begin{pmatrix}
                I_a & 0 \\
                A & T
            \end{pmatrix}.
        \]
        Here \( I_{a} \) is an \( a \times a \) identity matrix, \( A \) is the \(
        t \times a \) matrix of single-step transition probabilities from the \(
        t \) transient states to the \( a \) absorbing states, \( T \) is a \( t
        \times t \) submatrix of single-step transition probabilities among the
        transient states, and \( 0 \) is a \( a \times t \) matrix of \( 0 \)s
        representing the single-step transition probabilities from absorbing
        states to transient states.
    \item
        The matrix \( N = (I-T)^{-1} \) is the \defn{fundamental matrix}%
        \index{fundamental matrix}
        for the absorbing Markov chain.  The entries \( N_{ij} \) of
        this matrix have a probabilistic interpretation.  The entries \(
        N_{ij} \) are the expected number of times that the chain
        started from state \( i \) will be in state \( j \) before
        ultimate absorption.
    \item
        First-step analysis gives a compact expression in
        vector-matrix form for the waiting time \(
        \mathbf{w} \) to absorption:
        \[
            (I - T) \mathbf{w} = \mathbf{1}
        \] so \( \mathbf{w} = (I-T)^{-1} \mathbf{1} \).
\end{enumerate}

\hr

\visual{Vocabulary}{../../../../CommonInformation/Lessons/vocabulary.png}
\section*{Vocabulary}
\begin{enumerate}
    \item
        Let \( \{ X_n \} \) be a finite-state absorbing Markov chain with \( a \)
        absorbing states and \( t \) transient states.  Let the \( (a + t)
        \times (a + t) \) transition probability matrix be \( P \).  Order the
        states so the absorbing states come first and non-absorbing, i.e.\
        transient, states come last.  Then \( P^{(n)}_{ij} \to 0 \)
        as \( n \to \infty \) for \(i\) and \(j\) in the transient
        states, while for \( i\) in the absorbing states, \( P_{ii} =
        1 \).  Define the \defn{absorption time} \index{absorption
          time} as the random variable
        \[
            T = \min \setof{n \ge 0}{X_n \le a}.
        \]
    \item
        The \defn{absorption probability matrix} \index{absorption
            probability matrix} \( B \) is the
        probability of starting at state \( i \) and ending at a given
        absorbing state \( j \).
    \item
        For a Markov chain with \( a \) absorbing states and \( t \)
        transient states, if necessary, reorder the states so the
        absorbing states come first and non-absorbing, i.e.\ transient,
        states come last.  Then the transition matrix has the canonical
        form:
        \[
            P =
            \begin{pmatrix}
                I_a & 0 \\
                A & T
            \end{pmatrix}
            .
        \] The matrix \( N = (I-T)^{-1} \) is the \defn{fundamental
        matrix}%
        \index{fundamental matrix}
        for the absorbing Markov chain.
    \item
        The \( N \)th \defn{harmonic number} \( H_N \) is%
        \index{harmonic number}
        \[
            H_N = 1 + \frac{1}{2} + \frac{1}{3} + \frac{1}{4} 
                   + \frac{1}{5} + \cdots + \frac{1}{N}.
        \]
\end{enumerate}

\hr

\visual{Mathematical Ideas}{../../../../CommonInformation/Lessons/mathematicalideas.png}
\section*{Mathematical Ideas}

\subsection*{Theory}

Let \( \{ X_n \} \) be a finite-state absorbing Markov chain with \( a \)
absorbing states and \( t \) transient states.  Let the \( (a + t)
\times (a + t) \) transition probability matrix be \( P \).  Order the
states so the absorbing states come first and non-absorbing, i.e.\
transient, states come last.  The states
 \( a+1, 1, 2, \dots, a+t \) are transient in that
\( P^{(n)}_{ij} \to 0 \) as \( n \to \infty \) for
\( a+1 \le i,j \le a+t \), while states \( 1, \dots, a \) are absorbing,
\( P_{ii} = 1 \) for \( 1 \le i \le a \).
Then the transition probability matrix has the block-matrix form
\[
    P =
    \begin{pmatrix}
        I_a & 0 \\
        A & T
    \end{pmatrix}.
\]
Here \( I_{a} \) is an \( a \times a \) identity matrix, \( A \) is the \(
t \times a \) matrix of single-step transition probabilities from the \(
t \) transient states to the \( a \) absorbing states, \( T \) is a \( t
\times t \) submatrix of single-step transition probabilities among the
transient states, and \( 0 \) is a \( a \times t \) matrix of \( 0 \)s
representing the single-step transition probabilities from absorbing
states to transient states.

Starting at one of the transient states \( i \) where
\( a+1 \le i \le a+t \) such a process will remain in the transient states
for some duration.  Ultimately, the process gets trapped in one of the
absorbing states \( i = 1, \dots a \).  Before the Markov chain
transitions to one of the absorbing states, the number of times it
visits a transient state is a random variable.  Let \( Y_{ij} \)
denote the number of visits the system makes to transient state
\( j \) before reaching an absorbing state, given the system started
in transient state \( i \). Thus, \( Y_{ij} \) is a discrete random
variable that can take on any nonnegative integer value.  The random
variables \( Y_{ij} \) are the fundamental random variables of
interest here.  These fundamental random variables are the building blocks
for constructing and investigating other random variables.  The mean,
variance and covariances of the \( Y_{ij} \) are the first
statistics to investigate.  Of special interest is the mean time until
absorption.%
\index{mean time until absorption} 
Define the \defn{absorption time}%
\index{absorption time}
\[
    w_i = \min \setof{n \ge 0}{X_n \le a \given X_0 = i}.
\] Notice that \( w_i = \sum{k=a+1}^{a+t} Y_{ik} \),
the total number of visits the process makes among the transient
states.
The expected value of this random time \( \E{w_i} \), \( i = a+1, \dots,
a+t \) is a first measure of the random variable \( w_i \). Also of
interest is the probability distribution of the states into which
absorption takes place.  Using the fundamental random variables, it is
possible to compute this probability too.

\subsubsection*{Indicator Bernoulli Random Variables}

Let the indicator random variables be
\[
    U_{ij}^{(m)} =
    \begin{cases}
        1 & \text{if the Markov chain is in transient state \(j\)} \\
          & \text{after \(m\) steps given that it starts in transient state \(i\) } \\
        0 & \text{if the Markov chain is \emph{not} in transient state
          \(j\) } \\
          & \text{after \(m\) steps given that it starts in transient state \(i\)}
    \end{cases}
\] for \( m = 0, 1,2, \dots \).  The case \( m = 0 \) simply indicates
where the system starts:
\[
    U_{ij}^{(0)} =
    \begin{cases}
        1 & \text{if the Markov chain starts in transient state \(i\) }
        \\
        0 & \text{if the Markov chain does \emph{not} start in transient
        state \(i\). }
    \end{cases}
\] Using the usual notation, \( u_{ij}^{(0)} \) is the Kronecker delta function, \(
\delta_{ij} \).  The indicator random variables connect to the
fundamental random variables \( Y_{ij} \) through the sum
\[
    Y_{ij} = \sum\limits_{m=0}^{\infty} U_{ij}^{(m)}.
\]

\subsubsection*{Expected number of visits between states}

The expected number of visits to transient state \( j \) given that the
Markov chain starts in transient state \( i \) in terms of the indicator
ransom variable is
\[
    \E{Y_{ij}} = \E{\sum\limits_{m=0}^{\infty} U_{ij}^{(m)} } = \sum\limits_
    {m=0}^{\infty} \E{U_{ij}^{(m)}}.
\]

Use mathematical induction to show
\[
    P^{m} =
    \begin{pmatrix}
        I_a & 0 \\
        A & T
    \end{pmatrix}
    ^{m-1}
    \begin{pmatrix}
        I_a & 0 \\
        A & T
    \end{pmatrix}
    =
    \begin{pmatrix}
        I_a & 0 \\
        (I_t + T + T^2 + \cdots + T^{m-1})A & T^m
    \end{pmatrix}.
\]  (See the exercises.)  The
elements \( (P^{m})_{ij} \) of \( P^m \) are the \( m \)-step transition
probabilities between all states.  Since \( \E{U_{ij}^{(m)}} = p_{ij}^{%
(m)} \), so
\[
    \E{Y_{ij}} = \sum\limits_{m=0}^{\infty} \E{U_{ij}^{(m)}} = \sum\limits_
    {m=0}^{\infty} (P^{m})_{ij}}.
\] When \( i \) and \( j \) are transient states, then we only need to
consider the entries in the transient corner matrix \( T^m \), so
\[
    \E{Y_{ij}} = \sum\limits_{m=0}^{\infty} (T^m)_{ij} = \left( \sum\limits_
    {m=0}^{\infty} T^m \right)_{ij}.
\] The basic theory of finite absorbing Markov chains ensures that the
Euclidean norm of \( T \) is less than \( 1 \), \( \| T \| < 1
\). (See the exercises.)
Therefore \( \sum_{m=0}^{\infty} T^m \) converges.  Furthermore,
it converges to the \defn{fundamental matrix} \( N = (I-T)^{-1} \).%
\index{fundamental matrix}
Thus \( \E{Y_{ij}} = N_{ij} \).

\subsubsection*{Waiting time to absorption}

\begin{theorem}
    The entries \( N_{ij} \) of the fundamental matrix are the
    expected number of times that the chain started from state \( i \) will
    be in state \( j \) before ultimate absorption and the vector of expected
    waiting times to absorption is
    \( \E{\mathbf{w}} = N = (I-T)^{-1} \mathbf{1} \).
\end{theorem}

\begin{proof}
  The sum over all states $j$ of the number of times that the chain
  started from state \( i \) will be in state \( j \) before ultimate
  absorption is the waiting time to absorption.
\end{proof}

\subsubsection*{Covariances of numbers of visits between states }

Now use the indicator Bernoulli random variables to derive \(
\Cov{ Y_{ij} }{ Y_{ik} } \) where \( i \), \( j \), \( k \) are transient
states. Recall that
\[
    \Cov{ Y_{ij} }{ Y_{ik} } = \E{Y_{ij} \cdot Y_{ik}} - \E{Y_{ij}}
    \cdot \E{Y_{ik}}
\] and since \( \E{Y_{ij}} \) and \( \E{Y_{ik}} \) are known, all
that is necessary is \( \E{Y_{ij} \cdot Y_{ik}} \).  Start with
\[
    Y_{ij} Y_{ik} = \left( \sum\limits_{x=0}^{\infty} U_{ij}^{(x)}
    \right) \left( \sum\limits_{y=0}^{\infty} U_{ik}^{(y)} \right) =
    \sum\limits_{x=0}^{\infty} \sum\limits_{y=0}^{\infty} U_{ij}^{(x)} U_
    {ik}^{(y)}
\] so that
\[
    \E{Y_{ij} Y_{ik}} = \E{\sum\limits_{x=0}^{\infty} \sum\limits_{y=0}^
    {\infty} U_{ij}^{(x)} U_{ik}^{(y)}} = \sum\limits_{x=0}^{\infty}
    \sum\limits_{y=0}^{\infty} \E{U_{ij}^{(x)} U_{ik}^{(y)}}.
  \]
  Rearrange the double sum into a
   first term summing over the lattice points above the line \(
y=x \), a second term summing over the lattice points along the line \(
y=x \), and a third term summing over lattice points below the line \( y=x \).
\[
    \sum\limits_{x=0}^{\infty} \sum\limits_{y=x+1}^{\infty} \E{U_{ij}^{(x)}
    U_{ik}^{(y)}} + \sum\limits_{x=0}^{\infty} \E{U_{ij}^{(x)} U_{ik}^{(x)}}
    + \sum\limits_{y=0}^{\infty} \sum\limits_{x=y+1}^{\infty} \E{U_{ik}^
    {(y)} U_{ij}^{(x)}}.
\] Note that the
The first and third terms are symmetric, so only evaluate the first
term.

The expression \( \E{U_{ij}^{(x)} U_{ij}^{(y)}} \) is the probability
that the system is in transient state \( j \) after exactly \( x \)
steps from the start in state \( i \) \emph{and} the system is in transient
state \( k \) after exactly \( y \) steps from the start in state \( i
\).  Recall that \( x < y \).
Using the Markov chain property, this is \( \E{U_{ij}^{(x)} U_{ik}^{(y)}}
= p_{ij}^{(x)} p_{jk}^{(y-x)} \).  Therefore, the first term is
\begin{multline*}
    \sum\limits_{x=0}^{\infty} \sum\limits_{y=x+1}^{\infty} \E{U_{ij}^{(x)}
    U_{ik}^{(y)}} = \sum\limits_{x=0}^{\infty} \sum\limits_{y=x+1}^{\infty}
    p_{ij}^{(x)} p_{jk}^{(y-x)}
    = \sum\limits_{x=0}^{\infty} \sum\limits_{y-x=1}^{\infty} p_{ij}^{%
    (x)} p_{jk}^{(y-x)}
    = \sum\limits_{x=0}^{\infty} \sum\limits_{z=1}^{\infty} p_{ij}^{(x)}
    p_{jk}^{(z)}
    = \left( \sum\limits_{x=0}^{\infty} p_{ij}^{(x)} \right) \left(
    \sum\limits_{z=1}^{\infty} p_{jk}^{(z)} \right)
     = \left( \sum\limits_{x=0}^{\infty} p_{ij}^{(x)} \right) \left(
    \sum\limits_{z=0}^{\infty} p_{jk}^{(z)} - p_{jk}^{(0)}\right) \\
     = \left( \sum\limits_{x=0}^{\infty} (T^x){}_{ij} \right) \left(
    \sum\limits_{z=0}^{\infty} (T^z){}_{jk} - \delta_{jk}\right)
     = \left( \sum\limits_{x=0}^{\infty} T^x \right)_{ij} \left(
    \sum\limits_{z=0}^{\infty} T^z\right)_{jk} - \delta_{jk} \\
     = \left( (I_t-T)^{-1} \right)_{ij} \left( (I_t-T)^{-1} \right){}_{jk}
    - \delta_{jk} \\
    = N_{ij}(N_{jk}-\delta_{jk}).
\end{multline*}
The third term is the first term with \( j \) and \( k \) interchanged,
so
\[
    \sum\limits_{y=0}^{\infty} \sum\limits_{x=y+1}^{\infty} \E{U_{ik}^{(y)}
    U_{ij}^{(x)}} = N_{ik}(N_{kj}-\delta_{kj}).
\] Finally, the second term is \( \sum\limits_{x=0}^{\infty} \E{U_{ij}^{(x)}
U_{ik}^{(x)}} \) where each summand is the probability that the Markov
chain is in transient state \( j \) after exactly \( x \) steps after
starting in transient state \( i \) and simultaneously in state \( k \)
after exactly \( x \) steps starting from transient state \( i \).  This
is only possible if \( j=k \) hence \( \E{U_{ij}^{(x)} U_{ik}^{(x)}} = p_{ij}^{(x)}
\delta_{jk} \).  Thus for the second term
\[
    \sum\limits_{x=0}^{\infty} \E{U_{ij}^{(x)} U_{ik}^{(x)}} = \sum\limits_{x=0}^
    {\infty} p_{ij}^{(x)} \delta_{jk} = \left( \sum\limits_{x=0}^{\infty} (T^
    {x}){}_{ij} \right) \delta_{jk} = \left( \sum\limits_{x=0}^{\infty}
    T^{x} \right)_{ij} \delta_{jk} = N_{ij} \delta_{jk}.
\] Putting all terms together
\[
    \E{Y_{ij} Y_{ik}} = N_{ij}(N_{jk}-\delta_{jk}) + N_{ij} \delta_{jk}
    + N_{ik}(N_{kj}-\delta_{kj}) = N_{ij}N_{jk} + N_{ik}N_{kj} - N_{ik}\delta_
    {kj}.
\] Then
\begin{equation}
    \Cov{Y_{ij}}{Y_{ik}} = \E{Y_{ij} Y_{ik}} - \E{Y_{ij}} \E{Y_{ik}} = N_
    {ij}N_{jk} + N_{ik}N_{kj} - N_{ij}N_{ik} - N_{ik}\delta_{kj}.
     \label{eq:newwaitingtime:cov}
\end{equation}
Note that the result is symmetric under the interchange of \( j \) and
\( k \) as it should be.

In particular, letting \( j=k \) we get the variance of the number of
visits to state \( j \) starting from \( i \):
\[
    \Var{Y_{ij}} = 2 N_{ij}N_{jj} - N_{ij}^2 - N_{ij}.
\]  Note that this is the variance of the number of visits to transient
state \( j \) when starting from state \( i \), \emph{not} the variance of
the waiting time.
Define \(
\operatorname{diag}(N) \) to be the diagonal matrix setting all
off-diagonal elements of \( N \) to \( 0 \), and define \( N_{\text{sq}}
\) to be the matrix resulting from squaring each entry, then
\[
    \Var{Y_{(i)}} = N(2
    \operatorname{diag}
    (N) - I) - N_{\text{sq}}.
\]

The covariance, for a fixed \( i \), is the \( t \times t \)
matrix of entries \( \Cov{Y_{ij}}{ Y_{ik}} \).    Letting \( N_{(i,\cdot)} \) denote the \( 1
\times t \) row-vector from row \( i \) and setting \( (\operatorname{diag}
    N_{(i,\cdot)}) \) to be the diagonal matrix with this vector along the
    diagonal this becomes
\[
    \Cov{Y_{(i)}} =
    (\operatorname{diag}
    {N_{(i,\cdot)}}) N + N^T
    (\operatorname{diag}
    {N_{(i,\cdot)}}) - N_{(i,\cdot)} N_{(i,\cdot)}^T -
    (\operatorname{diag}
    N_{(i,\cdot)}.
\]  Note the outer product \( N_{(i,\cdot)} N_{(i,\cdot)}^T \).
The variance of the waiting time until absorption into any state from state \( i \)
is then the sum of all \( t^2 \) entries in \( \Var{Y_{(i)}} \).

\subsection*{First-step analysis}

First-step analysis%
\index{first-step analysis}
says the absorption time from state \( i \) is the first step
to another transient state plus a weighted average, according to the
transition probabilities over the transient states, of the absorption
times from the other transient states.  In symbols, first-step analysis
says
\[
    w_i = 1 + \sum\limits_{j=a+1}^{a+t} P_{ij} w_j.
\]

% Let \( \indicator{j}{Y_{n}} \) be the \defn{indicator function} for state
% \( j \) of the Markov chain \( Y_{n} \).  Recall that this means \(
% \indicator{j}{X_{n}} = 1 \) if \( X_{n} = j \) and \( \indicator{j}{X_{n}}
% = 0 \) otherwise.

As before, for a Markov chain with \( a \) absorbing states and \( t \)
transient states, reorder the states so the absorbing states come first
and non-absorbing, i.e.\ transient, states come last. Then the
transition matrix has the canonical form:
\[
    P =
    \begin{pmatrix}
        I_a & 0 \\
        A & T
    \end{pmatrix}
    .
\] Here \( I_{a} \) is an \( a \times a \) identity matrix, \( A \) is
the \( t \times a \) matrix of single-step transition probabilities from
the \( t \) transient states to the \( a \) absorbing states, \( T \) is
a \( t \times t \) submatrix of single-step transition probabilities
among the transient states, and \( 0 \) is a \( a \times t \) matrix of \(
0 \)s representing the single-step transition probabilities from
absorbing states to transient states.

Expressing the first-step analysis compactly in vector-matrix form as
\[
    \mathbf{w} = \mathbf{1} + T \mathbf{w}
\] or
\[
    (I - T) \mathbf{w} = \mathbf{1}.
\] Then \( \mathbf{w} = (I-T)^{-1} \mathbf{1} \).  The matrix \( N = (I-T)^
{-1} \) is the \defn{fundamental matrix}%
\index{fundamental matrix}
for the absorbing Markov chain.  The entries \( N_{ij} \) of this matrix
have a probabilistic interpretation.  The entries \( N_{ij} \) are the
expected number of times that the chain started from state \( i \) will
be in state \( j \) before ultimate absorption.

The \( t \times a \) matrix of \defn{absorption probabilities}%
\index{absorption probability matrix}
\( B \) has as entries the probability of starting at state \( i \) and ending up
at a given absorbing state \( j \).  The absorption probabilities come
from \( N \) by the matrix product \( B = NA = (I-T)^{-1}A \).  
% Roughly,
% add up all the probabilities of going to state \( j \), weighted by the
% number of times we expect to be in the transient states.  More
% precisely, we have the following theorem.

\begin{theorem}
  Let \( b_{ij} \) be the probability of the Markov chain starting
  in transient state \( i \) and ending in absorbing state $j$, then
  \[
        B = ( b_{ij} ) = NA.
  \]
\end{theorem}

\begin{proof}
   The proof is by first-step analysis.  Starting in state \( i \),
   the process may be captured in \( j \) in one or more steps.  The
   probability of capture in a single step is \( p_{ij} \).  If this
   does not happen, the process may move to another absorbing state,
   in which case it is impossible to reach \( j \), or to a transient
   state \( k \).  In the latter case, the probability of being
   captured in \( j \) is \( b_{kj} \).  Hence
   \[
        b_{ij} = p_{ij} + \sum\limits_{k =a+1}^{a+t} p_{ik} b_{kj}
   \]
   or in matrix form \( B = A + TB\). Thus, \( B = (I-T)^{-1} A = NA \). 
\end{proof}

Kemeny and Snell \cite[page 51]{kemeny60} give an alternative expression for the variance of
the waiting time using first-step analysis.

\begin{theorem}
\[
     \Var{\mathbf{w}} = (2N - I)N\mathbf{1} -
     (N\mathbf{1})_{\text{sq}}
\]
where \( (N\mathbf{1})_{\text{sq}} \) is the element-wise squared vector.
\end{theorem}

\begin{proof}
  \begin{enumerate}
  \item 
  Recall that \( \E{\mathbf{w}} = N \mathbf{1} \).
  Then use first-step analysis to evaluate \( \E{\mathbf{w}_{sq}} \).
\item 
  From its starting state \( i \) it can go to any state \( k \) with
  probability $p_{ik}$.  If the new state is absorbing, then it can
  never reach another state, and the contribution is \( 1 \) step.  If
  the new state is transient count the weighted average of the squares
  of the waiting times plus \(1 \).
\item 
\begin{align*}
     \E{\mathbf{w}_{sq}} &=
     \sum\limits_{k=1}^{a} p_{ik} \mathbf{1} +
                         \sum\limits_{k=a+1}^{a+t} p_{ik}
                         (\E{(\mathbf{w}+\mathbf{1})_{\text{sq}} })_k \\
                         &= \sum\limits_{k=1}^{a} p_{ik} \mathbf{1} +
                           \sum\limits_{k=a+1}^{a+t} p_{ik}
                           \left( (\E{ \mathbf{w}}_{\text{sq}})_k + 2
       (\E{ \mathbf{w}})_k + \mathbf{1} \right) \\
     &= T \E{\mathbf{w}_{\text{sq}}} + 2 T\E{\mathbf{w}} + \mathbf{1}.
\end{align*}
\item 
   Rearrange using \( N = (I_t - T)^{-1} \) and \( NT =
   (I_t - T)^{-1} T = N - I \) (See the exercises.) 
\begin{align*}
     \E{\mathbf{w}_{sq}} &= (I_t - T)^{-1} (2 T\E{\mathbf{w}} +
                           \mathbf{1}) \\
                        &= 2NT \E{\mathbf{w}} + N \mathbf{1} \\
                        &= 2(N-I) \E{\mathbf{w}} + \E{\mathbf{w}} \\
                        &= (2N-I) \E{\mathbf{w}}.
\end{align*}
\item 
Thus
\[
  \Var{\mathbf{w}} = (2N - I)N\mathbf{1} -
     (N\mathbf{1})_{\text{sq}}.
\]
  \end{enumerate}
\end{proof}

\begin{remark}
  Carchidi and Higgins~\cite{carchidi17} give an alternative proof and slightly
  different derivation of this formula for the variance of the waiting
  time until absorption using equation~\eqref{eq:newwaitingtime:cov}.
\end{remark}
\subsection*{Examples}

\begin{example}
    Consider a random walk%
    \index{random walk}
    of a particle which moves along a straight line in unit steps.  Each
    step is \( 1 \) unit to the right with probability \( p \) and to
    the left with probability \( q \).  It moves until it reaches one of
    two extreme points which are \defn{absorbing
      boundaries}. \index{absorption probability matrix}   Assume that if
    the process reaches the boundary points, it remains there from that
    time on. Figure~%
    \ref{fig:waitingtimeabsorbtion:randomwalkphasespace} has \( 9 \)
    states numbered from \( -4 \) to \( 4 \).  The absorbing boundary
    states are \( -4 \) and \( 4 \).

    \begin{figure}
        \centering
        \begin{asy}
        import graph;

        size(5inches);

        real myfontsize = 10;
        real mylineskip = 1.2*myfontsize;
        pen mypen = fontsize(myfontsize, mylineskip);
        defaultpen(mypen);

        xaxis(xmin=-5, xmax=5,
              ticks=RightTicks(beginlabel=false, endlabel=false, N=10, Step=0));
        draw( (-3,0)--(3,0), linewidth(2pt));
        dot( (3,0) );

        draw( (0,1/10)--(1,1/10), arrow=Arrow(TeXHead));

        draw( (1,2/10)--(0,2/10), arrow=Arrow(TeXHead));
        draw((0,2/10)--(-1,2/10),arrow=Arrow(TeXHead));
        draw((-1,2/10)--(-2,2/10),arrow=Arrow(TeXHead));
        draw((-2,2/10)--(-3,2/10),arrow=Arrow(TeXHead));

        draw( (-3,3/10)--(-2,3/10), arrow=Arrow(TeXHead));

        draw((-2,4/10)--(-3,4/10),arrow=Arrow(TeXHead));

        draw((-3,5/10)--(-2,5/10),arrow=Arrow(TeXHead));

        draw((-2,6/10)--(-3,6/10),arrow=Arrow(TeXHead));

        draw((-3,7/10)--(-2,7/10),arrow=Arrow(TeXHead));
        draw((-2,7/10)--(-1,7/10),arrow=Arrow(TeXHead));
        draw((-1,7/10)--(0,7/10),arrow=Arrow(TeXHead));

        draw((0,8/10)--(-1,8/10),arrow=Arrow(TeXHead));

        draw((-1,9/10)--(0,9/10),arrow=Arrow(TeXHead));
        draw((0,9/10)--(1,9/10),arrow=Arrow(TeXHead));
        draw((1,9/10)--(2,9/10),arrow=Arrow(TeXHead));
        draw((2,9/10)--(3,9/10),arrow=Arrow(TeXHead));
        \end{asy}
        %% \includegraphics{randomwalkphasespace.png}
        \caption{Image of a possible random walk in phase line after an
        odd number of steps.}%
        \label{fig:waitingtimeabsorbtion:randomwalkphasespace}
    \end{figure}

    The full transition probability matrix is
    \[
        P =
        \bordermatrix{
              & -4 & -3 & -2 & -1 & 0 & 1 & 2 & 3 & 4 \cr
           -4 &  1 &  0 &  0 &  0 & 0 & 0 & 0 & 0 & 0 \cr
           -3 &  q &  0 &  p &  0 & 0 & 0 & 0 & 0 & 0 \cr
           -2 &  0 &  q &  0 &  p & 0 & 0 & 0 & 0 & 0 \cr
           -1 &  0 &  0 &  q &  0 & p & 0 & 0 & 0 & 0 \cr
            0 &  0 &  0 &  0 &  q & 0 & p & 0 & 0 & 0 \cr
            1 &  0 &  0 &  0 &  0 & q & 0 & p & 0 & 0 \cr
            2 &  0 &  0 &  0 &  0 & 0 & q & 0 & p & 0 \cr
            3 &  0 &  0 &  0 &  0 & 0 & 0 & q & 0 & p \cr
            4 &  0 &  0 &  0 &  0 & 0 & 0 & 0 & 0 & 1 \cr
        }
    \]

    Reorder the states as \( -4, 4, -3, -2, -1, 0, 1, 2, 3 \) to bring
    the transition probability matrix to the standard form
    \[
        P_{1} =
        \begin{pmatrix}
            1 & 0 & 0 & 0 & 0 & 0 & 0 & 0 & 0 \\
            0 & 1 & 0 & 0 & 0 & 0 & 0 & 0 & 0 \\
            q & 0 & 0 & p & 0 & 0 & 0 & 0 & 0 \\
            0 & 0 & q & 0 & p & 0 & 0 & 0 & 0 \\
            0 & 0 & 0 & q & 0 & p & 0 & 0 & 0 \\
            0 & 0 & 0 & 0 & q & 0 & p & 0 & 0 \\
            0 & 0 & 0 & 0 & 0 & q & 0 & p & 0 \\
            0 & 0 & 0 & 0 & 0 & 0 & q & 0 & p \\
            0 & p & 0 & 0 & 0 & 0 & 0 & q & 0 \\
        \end{pmatrix}
    \] so
    \[
        T =
        \begin{pmatrix}
            0 & p & 0 & 0 & 0 & 0 & 0  \\
            q & 0 & p & 0 & 0 & 0 & 0  \\
            0 & q & 0 & p & 0 & 0 & 0  \\
            0 & 0 & q & 0 & p & 0 & 0  \\
            0 & 0 & 0 & q & 0 & p & 0  \\
            0 & 0 & 0 & 0 & q & 0 & p  \\
            0 & 0 & 0 & 0 & 0 & q & 0  \\
        \end{pmatrix}.
    \] A computer algebra system such as Maxima for example can compute
    the fundamental matrix but the general form in terms of \( p \)
    and \( q \) is long, messy and unhelpful.
    Two representative numerical examples suffice to show
    the possibilities.

    \[
        N(1/2) =
        \begin{pmatrix}
            7/4 & 3/2 & 5/4 & 1 & 3/4 & 1/2 & 1/4\\
            3/2 & 3 & 5/2 & 2 & 3/2 & 1 & 1/2\\
            5/4 & 5/2 & 15/4 & 3 & 9/4 & 3/2 & 3/4\\
            1 & 2 & 3 & 4 & 3 & 2 & 1\\
            3/4 & 3/2 & 9/4 & 3 & 15/4 & 5/2 & 5/4\\
            1/2 & 1 & 3/2 & 2 & 5/2 & 3 & 3/2 \\
            1/4 & 1/2 & 3/4 & 1 & 5/4 & 3/2 & 7/4
        \end{pmatrix}
    \] Then the waiting times to absorption are \( 7, 12, 15, 16, 15,
    12, 7 \).  For the variances, consider only the central state,
    originally labeled as \( 0 \), after reordering it is the sixth
    state in the middle of the transient states.
    \[
        \Cov{Y_{0j}}{ Y_{0k}} =
        \begin{pmatrix}
            3/2 & 5/2 & 2 & 1 & 0 & -1/2 & -1/2 \\
            5/2 & 6 & 13/2 & 4 & 3/2 & 0 & -1/2\\
            2 & 13/2 & 21/2 & 9 & 9/2 & 3/2 & 0\\
            1 & 4 & 9 & 12 & 9 & 4 & 1\\
            0 & 3/2 & 9/2 & 9 & 21/2 & 13/2 & 2\\
            -1/2 & 0 & 3/2 & 4 & 13/2 & 6 & 5/2\\
            -1/2 & -1/2 & 0 & 1 & 2 & 5/2 & 3/2
        \end{pmatrix}
    \] The variance of the number of visits to \( 0 \) until
    absorption is \( \mathbf{1}^{T} V \mathbf{1} = 160 \), the
    standard deviation is \( 4 \sqrt{10} \).  The negative covariance
    \( -1/2 \)
    of the number of visits until absorption from \( 0 \) to \( -3 \)
    with the  number of visits until absorption from \( 0 \) to \( 3 \) 
    indicates that the random number of visits until absorption
    from \( 0 \) to \( -3 \) has a somewhat opposite distribution to
    that of the random number of visits until absorption
    from \( 0 \) to \( 3 \).  That is, when the number of visits until absorption
    from \( 0 \) to \( -3 \) is large, the number of visits until absorption
    from \( 0 \) to \( 3 \) will be small.  This makes sense, since if
    the number of visits from \( 0 \) to \( -3 \) is large, the
    likelihood of absorption into \( -4 \)  increases, and the
    number of visits from \( 0 \) to \( 3 \) decreases.

    The probabilities of absorption are
    \[
        N(1/2) A =
        \begin{pmatrix}
            7/8 & 1/8\\
            3/4 & 1/4\\
            5/8 & 3/8\\
            1/2 & 1/2\\
            3/8 & 5/8\\
            1/4 & 3/4\\
            1/8 & 7/8
        \end{pmatrix}
    \] which are symmetric as expected.

    If \( p = 2/3 \) so the probability of moving to the right is twice
    the probability of moving to the left, then
    \[
        N(2/3) =
        \begin{pmatrix}
            127/85 & 126/85 & 124/85 & 24/17 & 112/85 & 96/85 & 64/85\\
            63/85 & 189/85 & 186/85 & 36/17 & 168/85 & 144/85 & 96/85\\
            31/85 & 93/85 & 217/85 & 42/17 & 196/85 & 168/85 & 112/85\\
            3/17 & 9/17 & 21/17 & 45/17 & 42/17 & 36/17 & 24/17\\
            7/85 & 21/85 & 49/85 & 21/17 & 217/85 & 186/85 & 124/85\\
            3/85 & 9/85 & 21/85 & 9/17 & 93/85 & 189/85 & 126/85\\
            1/85 & 3/85 & 7/85 & 3/17 & 31/85 & 63/85 & 127/85
        \end{pmatrix}
    \] and the waiting times to absorption are approximately \( 9.05 \),
    \( 12.07 \), \(12.08\), \(10.59\),
    \(8.34\), \(5.72\), \(2.91\). For the
    variances, consider only the central state, originally labeled as \(
    0 \) and after reordering is the sixth state in the middle of the
    transient states.
    \[
        \Cov{Y_{0j}}{Y_{0k}} =
        \begin{pmatrix}
        462/1445 & 81/85 & 2661/1445 & 1017/289 & 882/289 &
          3204/1445 & 1368/1445\\ 
        621/1445 & 2232/1445 & 927/289 & 1539/289 & 6678/1445 & 972/289 &
        2088/1445\\ 
        429/1445 & 369/289 & 5124/1445 & 1827/289 & 7938/1445 & 5796/1445 &
        504/289\\
        9/289 & 81/289 & 441/289 & 1260/289 & 1764/289 & 1296/289 & 576/289\\ 
        -63/289 & -819/1445 & -1323/1445 & -63/289 & 5838/1445 & 6948/1445 &
        3144/1445\\ 
        -423/1445 & -243/289 & -2583/1445 & -891/289 & -18/85 & 4068/1445 &
        648/289\\ 
        -333/1445 & -981/1445 & -441/289 & -873/289 & -2994/1445 & -108/289 &
        1176/1445
        \end{pmatrix}
    \] The variance is \( \mathbf{1}^{T} V \mathbf{1} = 21708/289
    \approx 75.11418 \), the standard deviation is \( 8.66684 \), less
    than the symmetric case where \( p =1/2 = q \), as expected.

    The probabilities of absorption are
    \[
        N(2/3) A =
        \begin{pmatrix}
            127/255 & 128/255\\
            21/85 & 64/85\\
            31/255 & 224/255\\
            1/17 & 16/17\\
            7/255 & 248/255\\
            1/85 & 84/85\\
            1/255 & 254/255
        \end{pmatrix}.
    \] The absorption probabilities are strongly biased to the right, as
    expected.
\end{example}

\begin{example}
    A law firm employs three types of lawyers:  junior lawyers, senior
    lawyers, and partners.  During a given year, there is probability \(
    0.15 \) of promoting a junior lawyer to senior lawyer and
    probability \( 0.5 \) that the junior lawyer will leave the firm.
    There is probability \( 0.20 \) of promoting the senior lawyer to
    partner and probability \( 0.10 \) that the senior lawyer will leave
    the firm.  Finally, there is probability \( 0.05 \) that a partner
    will leave the firm, see
    Table~\ref{tab:newwaitingtimeabsorption:lawyers}.  
    The firm never demotes a lawyer or a partner.
    What is the average number of years that a newly hired junior lawyer
    stays with the firm?

    \begin{table}[htbp]
        \caption[]{The transition probabilities }
        \vspace{4mm}
        \begin{tabular}[tb]
            {r|cccc} & leave & junior & senior & \\
            & firm & lawyer & lawyer & partner \\
            \hline
            leave firm & 1 & 0 & 0 & 0 \\
            junior lawyer & 0.05 & 0.80 & 0.15 & 0 \\
            senior lawyer & 0.10 & 0 & 0.70 & 0.20 \\
            partner & 0.05 & 0 & 0 & 0.95
        \end{tabular}
        \label{tab:newwaitingtimeabsorption:lawyers}
    \end{table}

This leads to the single-step transition matrix \( P \) partitioned into
one absorbing state for ``leave firm'' and three transient
states: junior lawyer, senior lawyer, partner
\[
    P =
    \begin{pmatrix}
        1    & 0    & 0    & 0    \\
        0.05 & 0.80 & 0.15 & 0    \\
        0.10 & 0    & 0.70 & 0.20 \\
        0.05 & 0    & 0    & 0.95
    \end{pmatrix}.
\] The transient transition matrix is
\[
    T = 
    \begin{pmatrix}
         0.80 & 0.15 & 0 \\
               0 & 0.70 & 0.20 \\
               0 &    0 & 0.95
    \end{pmatrix}.
\] The corresponding fundamental matrix is
\[
    N= (I_3 - T)^{-1} =
    \begin{pmatrix}
        5 & 5/2 & 10 \\
        0 & 10/3 & 40/3 \\
        0 & 0 & 20
    \end{pmatrix}.
\] The waiting times to absorption are then
\[
    (I-T)^{-1} \mathbf{1} = (35/2, 50/3, 20)^{T}.
\]

Now compute the covariances for the first transient state, junior
lawyer.  Using
\[
    \operatorname{diag}
    {N_{(1)}} N + N^T
    \operatorname{diag}
    {N_{(1)}} - N_{(1)}^T N_{(1)} -
    \operatorname{diag}
    N_{(1)}
\] we get
\[
    \Cov{Y_{1j}}{Y_{1k}} =
    \begin{pmatrix}
        20 & 0 & 0 \\
        0 & 95/12 & 25/3 \\
        0 & 25/3 & 290
    \end{pmatrix}
    =
    \begin{pmatrix}
        20 & 0 & 0 \\
        0 & 7.91667 & 8.33333 \\
        0 & 8.33333 & 290
    \end{pmatrix}.
\]

Since \( w_i = \sum\limits_{j=a+1}^{t} Y_{ij} \), using that \( \Var{w_i}
= \sum\limits_{j=a+1}^t \sum\limits_{k=a+1}^t \Cov{Y_{1j}}{Y_{1k}} \),
the variance is the sum of the entries in the covariance matrix. Hence \(
\Var{w_1} = 334.58 \) (in units of \( \text{year}^2 \)).  The standard
deviation is \( 18.29 \) years.  As a point of special interest, note that the
standard deviation is larger than the mean so that knowing the mean is
not sufficient to understand how long a junior lawyer will be with the firm.
\end{example}

\begin{example}

    It’s your \( 30 \)th birthday, and your friends bought you a
    cake with \( 30 \) candles on it.  You make a wish and try to blow
    them out. Every time you blow, you blow out a random number of
    candles between one and the number that remain, including one and
    that other number.  How many times on average do you blow before you
    extinguish all the candles?

    Let the states be the number of candles remaining lit.  Order the
    \( N= 31 \) states as \( 0, 1, 2, 3, \dots, 29, 30 \).  Then the
    state \( 0 \) is absorbing and all other states are transient.
    Instead of solving this full problem at once, try a smaller
    problem first, with the number of candles \( N = 5 \). Then the
    transition probability matrix in canonical form is
    \[
        P =
        \begin{pmatrix}
            1 & 0 & 0 & 0 & 0 & 0 \\
            1 & 0 & 0  & 0 & 0 & 0 \\
            1/2 & 1/2 & 0 & 0 & 0 & 0 \\
            1/3 & 1/3 & 1/3 & 0 & 0 & 0 \\
            1/4 & 1/4 & 1/4 & 1/4 & 0 & 0 \\
            1/5 & 1/5 & 1/5 & 1/5 & 1/5 & 0 \\
        \end{pmatrix}.
    \] Then the first-step equations for the waiting time, that is the number of
    attempts needed to blow out the candles are
    \begin{align*}
        w_1 &= 1
        w_2 &= 1 + \frac{1}{2} w_1 \\
        w_3 &= 1 + \frac{1}{3} w_1 + \frac{1}{3} w_2 \\
        w_4 &= 1 + \frac{1}{4} w_1 + \frac{1}{4} w_2 + \frac{1}{4} w_3 \\
      w_5 &= 1 + \frac{1}{5} w_1 + \frac{1}{5} w_2 + \frac{1}{5} w_3 + \frac{1}{5} w_4\\
    \end{align*}
    Solving this recursively, \( w_1 = 1 \), \( w_2 = 1 + \frac{1}{2} \),
    \( w_{3} = 1 + \frac{1}{3} + \frac{1}{3} \cdot (1+ \frac{1}{2})  \), so \(
    w_3 = 1 + \frac{1}{2} + \frac{1}{3} \).  Continuing to solve
    recursively
    \begin{align*}
        w_4 &= 1 + 1/2 + 1/3 + 1/4 \\
        w_5 &= 1 + 1/2 + 1/3 + 1/4 + 1/5.
    \end{align*}
    The inductive pattern is clear.  The waiting time with \( N \)
    candles is the \( N \)th \defn{harmonic number} \( H_N \)%
    \index{harmonic number}
    \[
        w_N = H_N = 1 + \frac{1}{2} + \frac{1}{3} + \frac{1}{4} 
              + \frac{1}{5} + \cdots + \frac{1}{N}.
    \]

    For the original problem with \( 30 \) candles the transient states
    are \( 1, 2, 3, \dots, 30 \) and the absorbing state is \( 0 \).
    Then the canonical form transition probability matrix is
    \[
        P =
        \begin{pmatrix}
            1 & 0 & 0 & \ldots & 0 & 0 \\            
            1 & 0 & 0 & \ldots & 0 & 0 \\
            1/2 & 1/2 & 0 & \ldots & 0 & 0 \\
            1/3 & 1/3 & 1/3 & \ldots & 0 & 0 \\
            \vdots & \vdots & \vdots & \ddots & \vdots & \vdots \\
            1/30 & 1/30 & 1/30 & \ldots & 1/30 & 0 \\
        \end{pmatrix}
    \]

    The quantity of interest is the expected waiting time until
    absorption into the state \( 0 \).  Expressing the first-step
    analysis compactly in vector-matrix form as
    \[
        (I - T) \mathbf{W} = \mathbf{1}
    \] Substituting in the values from the transition matrix and solving
    with Octave, this is \( w_{30} = H_{30} \approx 3.9950 \).
    By calculating the covariance matrix and taking the sum of all
    entries,  the variance of the waiting time from the
    transient state of all candles lit is \( 2.3828 \).
\end{example}

\begin{example}
    An urn contains two unpainted balls.  At a sequence of times, choose
    a ball at random, then paint it either red or black, and put it back.
    For an unpainted ball, choose a color at random.  For a painted
    ball, change its color.  Form a Markov chain by taking as a state
    the triple \( (x,y,z) \) where \( x \) is the number of unpainted
    balls, \( y \) the number of red balls, and \( z \) the number of
    black balls.  The transition matrix is then
    \[
        \bordermatrix{
                    & (0,1,1) & (0,2,0) & (0,0,2) & (2,0,0) & (1,1,0) & (1,0,1)
            \cr
            (0,1,1) & 0 & 1/2 & 1/2 & 0 & 0 & 0 \cr
            (0,2,0) & 1 & 0 & 0 & 0 & 0 & 0 \cr
            (0,0,2) & 1 & 0 & 0 & 0 & 0 & 0 \cr
            (2,0,0) & 0 & 0 & 0 & 0 & 1/2 & 1/2 \cr
            (1,1,0) & 1/4 & 1/4 & 0 & 0 & 0 & 1/2 \cr
            (1,0,1) & 1/4 & 0 & 1/4 & 0 & 1/2 & 0 \cr
        }
    \] In this case, there is no absorbing state, that is, a state which
    once entered remains the same thereafter.  However, the first three
    states together are an ergodic set, that is, once the process enters
    that set, it continues in that set.  So lump those states together
    as a single absorbing state, and the transient states are \( (2,0,0)
    \), \( (1,1,0) \), and \( (1,0,1) \).  Then
    \[
        T =
        \begin{pmatrix}
            & 0 & 1/2 & 1/2 \\
            & 0 & 0 & 1/2 \\
            & 0 & 1/2 & 0 \\
        \end{pmatrix}
    \] and the fundamental matrix is
    \[
        N =
        \begin{pmatrix}
            1 & 1 & 1\\
            0 & 4/3 & 2/3\\
            0 & 2/3 & 4/3
        \end{pmatrix}
    \] With this, we can compute
    \[
        \E{\mathbf{w}} = N \mathbf{1} = (3, 2, 2)^T
    \] and
    \[
        \Var{\mathbf{w}} = (2N-I)\E{\mathbf{w}}-(\E{\mathbf{w}})_{\text{sq}} = (2,2,2)^T
    \] Finally,
    \[
        \Var{\mathbf{X}} = N(2
        \operatorname{diag}
        (N) - I) - N_{\text{sq}} =
        \begin{pmatrix}
            0 & 2/3 & 2/3 \\
            0 & 4/9 & 2/3 \\
            0 & 2/3 & 4/9
        \end{pmatrix}
    \] Since the process must immediately leave state \( (2,0,0) \) and
    cannot go back, there is \( 0 \) variance for the number of times in
    this state.
\end{example}

\begin{example}
    The following is a larger example of the painting the
    balls puzzle.

    You play a game with four balls in a box:  One ball is red, one is
    blue, one is green and one is yellow.  You draw a ball out of the
    box at random and note its color.  Without replacing the first ball,
    you draw a second ball and then paint it to match the color of the
    first.  Replace both balls, and repeat the process.  The game ends
    when all four balls have become the same color.  What is the
    expected number of turns to finish the game?  

    Take the states as the number of balls of each different color
    without regard for the colors themselves.  For example, partition 4
    balls in 5 different ways:
    \[
        1+1+1+1, \quad 2+1+1, \quad 2+2, \quad 3+1, \quad 4.
    \] The partition \( 2+1+1 \), for example, consists of cases where
    two of the balls are the same color and the other two balls are two
    other colors.  For example, the cases ``red \& red \& green \& blue''
    and ``blue \& blue \& yellow \& red' correspond to the partition \(
    2+1+1 \).  By using these five partitions as states in a Markov
    Chain, we can compute the transition probabilities to go from one
    state to the next.  For example, the probability of transitioning
    from \( 2+1+1 \) to \( 3+1 \) is \( 1/3 \) because in order
    for this transition to occur, we must first choose one of the two
    identically colored balls with probability \( 2/4 \), then
    we must choose one of the other two balls out of the remaining three
    with probability \( 2/3 \).  The joint probability is \(
    2/4 \cdot 2/3 = 1/3 \).  As another example,
    the probability of transitioning from \( 2 + 2 \) to \( 3 + 1 \) is
    the probability of first picking a ball of either color, leaving \(
    1 \) ball of that color and \( 2 \) balls of the other color, then
    from those \( 3 \) balls picking the second color with probability \(
    2/3 \).  Calculate all remaining transition probabilities in
    the same way.

    \[
        P =
        \bordermatrix{
            & 1+1+1+1 & 2+1+1 & 3 + 1 & 2 + 2 & 4 \cr
            1+1+1+1 & 0 & 1 & 0 & 0 & 0 \cr
            2+1+1 & 0 & 1/2 & 1/3 & 1/6 & 0 \cr
            3+1 & 0 & 0 & 1/2 & 1/4 & 1/4 \cr
            2+2 & 0 & 0 & 2/3 & 1/3 & 0 \cr
            4 & 0 & 0 & 0 & 0 & 1 \cr
        }
    \]

    The absorbing state is \( 4 \).  Rearranging into the canonical
    block-matrix form
    \[
        P =
        \bordermatrix{
            & 4 & 1+1+1+1 & 2+1+1 & 3 + 1 & 2 + 2 \cr
            4 & 1 & 0 & 0 & 0 & 0 \cr
            1+1+1+1 & 0 & 0 & 1 & 0 & 0 \cr
            2+1+1 & 0 & 0 & 1/2 & 1/3 & 1/6 \cr
            3+1 & 1/4 & 0 & 0 & 1/2 & 1/4 \cr
            2+2 & 0 & 0 & 0 & 2/3 & 1/3 \cr
        }
    \]

    Then \( N = (I-T)^{-1} \) is
    \[
        \begin{pmatrix}
            1 & 2 & 4 & 2\\
            0 & 2 & 4 & 2\\
            0 & 0 & 4 & 3/2\\
            0 & 0 & 4 & 3
        \end{pmatrix}
    \] and the waiting time from the initial state is \( 1 + 2 + 4 + 2 =
    9\).   From the state \( 1+1+1+1 \) (labeled as state \( 2 \) in the
    canonical format), the covariance matrix is
    \[
        \begin{pmatrix}
            0 & 0 & 0  & 0\\
            0 & 2 & 0  & 0\\
            0 & 0 & 12 & 6\\
            0 & 0 & 6  & 6
        \end{pmatrix}
    \] Then the variance of the number of visits is \( 32 \)  and the
    standard deviation of the number of visits is \( 4 \sqrt{2}
    \approx 5.66 \). 

\end{example}

\begin{example}
  The following example is from the December 22, 2017 Riddler at
  \link{https://fivethirtyeight.com/features/hark-two-holiday-puzzles/}{Fivethirtyeight.com}.
  It concerns a game of chance called Left, Right, Center. In this
  game, everyone sits in a circle and starts with some number of $\$1$
  bills. You take turns, in order around the circle, rolling three
  dice. For each die, if it comes up $1$ or $2$, you give a dollar to
  the person on your left. If it comes up $3$ or $4$, you give a
  dollar to the person on your right. And if it comes up $5$ or $6$,
  you put a dollar in the center.  Assume the following: First, if a
  player has no dollars, then her turn is skipped. Second, if a player
  has one or two dollars, then the player rolls only one or two dice,
  respectively. The game ends as soon as only a single person has any
  money left. How long is the game expected to last for six players
  each starting with three $\$1$ bills? For $X$ players each starting
  with $Y$ $\$1$ bills?

  Consider the allocation of the total of $\$18$ among the $6$ players
  and the center, that is $7$ places, as the states of a Markov
  chain.  Using the ``stars and bars'' argument, choosing $7-1 = 6$ bars
  from $18 + 6$ objects the number of states is
  $\binom{18+6}{6} = 134{,}596$.  This also counts all the money in the
  center as one state which would not strictly be a state of the
  game.  So there are $134{,}595$ actual game states.  This is the same as the number of
  states dividing any number of dollars from $1$ to $18$ among $6$
  players, ignoring the money in the center which would be
  $\sum\limits_{j=6}^{23}\binom{j}{5} = 134{,}595$.  This is essentially the
  conclusion of the hockey-stick identity for binomial coefficients.

  The game ends when exactly $1$ of the $6$ players has any amount of
  money from $\$1$ to $\$18$.  That is, there are $6 \cdot (1 + \cdots
  + 18) = 6 \cdot 18 \cdot 19/2 = 1026$ terminal states which can be
  considered as the absorbing states of the game.

  The shortest possible game would be for each player in turn to throw
  all $5$'s and $6$'s, thereby placing all the player's starting money
  in the center.  This shortest game would be $6$ turns with
  probability $(1/27)^6$.  Consider the
  total number of dollars in the circle at each turn.  The expected
  loss of dollars from the circle to the center at each turn is $1$.
  That means it takes should take on the order of $17$ turns to end
  the game.  More precisely, consider the lumped system where each
  state is the number of dollars in the circle, from $\$18$ to $\$1$.
  Let $D_j$ be the duration of the game from $j$ dollars with
  $j = 1, \dots, 18$.  Then a first-step analysis gives the system of
  equations $D_j = (8/27)D_j + (12/27)D_{j-1} + (6/27)D_{j-2} +
  (1/27)D_{j-3} + 1$, with $D_j = 0$ for $j = 1, 0, -1$.  The solution
  of this system with Octave gives $D_{18} = 17.3333$ which is
  consistent with the previous crude estimate.

  The number of states and absorbing states are too large to be
  effectively handled by matrix methods.  Although the conceptual
  set-up of the game is clearly the waiting time until absorbtion in a
  Markov chain, simulation seems to be the best way to answer the
  question.  Results of simulating games with $6$ to $9$ players each
  starting with $3$ to $6$ dollars are in the tables.  The Riddler
  says that the game with $X$ players each starting with $Y$ dollars
  will last $2(X-2)Y$ turns but does not give a proof.
  The simulations give means which are consistent with this value.
  \begin{table}
    \centering
    \begin{tabular}{l | llll}
      &  6       & 7       & 8       & 9      \\
     \hline
     3 &  25.7096 & 32.3160 & 38.9660 & 45.3878 \\
     4 &  31.7890 & 39.7598 & 47.3006 & 55.3804 \\
     5 &  38.3244 & 47.4416 & 56.3536 & 65.4438 \\
     6 &  44.8394 & 54.9732 & 65.5688 & 75.3548
    \end{tabular}
    \caption{Mean number of turns in $5000$ simulations of the game
      with $X$ players (columns) each with $Y$ dollars (rows).}
    \label{tab:waitingtimeabsorbtion:lcrmean}
  \end{table}

  \begin{table}
    \centering
    \begin{tabular}{l | llll}
      &  6       & 7       & 8       & 9      \\
     \hline
     3&   6.546804 &  7.009249 &  7.564145 &  7.898221 \\
     4& 7.751729 &  8.176348 &  8.594918 &  9.099816 \\
     5&  8.989401 &  9.367077 &  9.878952 & 10.387840 \\
     6&  10.161804 & 10.468601 & 11.080443 & 11.654668
\end{tabular}
    \caption{Variance of number of turns in $5000$ simulations of the game
      with $X$ players each with $Y$ dollars.}
    \label{tab:waitingtimeabsorbtion:lcrmean}
  \end{table}


% V = [(8/27)*ones(17,1), (12/27)*ones(17,1), (6/27)*ones(17,1),
% (1/27)*ones(17,1)]
% A = spdiags(V, [0,1,2,3], eye(17))
% (eye(17) - A)\ones(17,1)

% ans =

%    17.3333
%    16.3333
%    15.3333
%    14.3333
%    13.3333
%    12.3333
%    11.3333
%    10.3333
%     9.3333
%     8.3333
%     7.3333
%     6.3334
%     5.3331
%     4.3338
%     3.3342
%     2.3186
%     1.4211

\end{example}

\subsection*{Sources} This section is adapted from Carchidi and
Higgins~\cite{carchidi17}.  The section on first-step analysis is
adapted from \booktitle{Finite Markov Chains} by Kemeny and Snell.
Other ideas are from \booktitle{An Introduction to Stochastic
  Modeling} by Taylor and Karlin and \booktitle {Random Walks and
  Electrical Networks} by Doyle and Snell.  The birthday candle
example is adapted from the January 13, 2017 ``Riddler'' problem from
the webpage \texttt{fivethirtyeight.com}.  The two colored balls
example is adapted from \booktitle{Finite Markov Chains} by Kemeny and
Snell. The four colored balls example is adapted from
http://www.laurentlessard.com/bookproofs/colorful-balls-puzzle/
The Left-Center-Right game example is from the December 22, 2017
  Riddler at
\link{https://fivethirtyeight.com/features/hark-two-holiday-puzzles/}{Fivethirtyeight.com}.

\nocite{kemeny60} \nocite{charchid17}
\nocite{taylor98-introd-stoch-model} \nocite{doyle84}

\hr

\visual{Algorithms, Scripts, Simulations}{../../../../CommonInformation/Lessons/computer.png}
\section*{Algorithms, Scripts, Simulations}

\begin{lstlisting}[language=R]
  M <- matrix(0, 4, 4)
V <- matrix(0, 4, 4)

for (player_count in c(6:9)) {
    for (bills in c(3:6)) {
## player_count <-  6
## bills <-  3
simulations <- 5000

game <- function(){

    players <- rep(bills, player_count)
    turns <- 1
    current_player <- 1

    player_index <- function(j) {
        ((j-1) %% player_count) + 1
    }
    
    turn <- function(player) {
        rolls <- sample(6, 3, replace=TRUE)
        for (roll in rolls) {
            if (players[player] > 0) {
                if (roll <= 2) {
                    players[player_index(player - 1)] =
                        players[player_index(player - 1)] + 1 
                    players[player] = players[player] - 1
                } else if (roll <= 4) {
                    players[player_index(player + 1)] =
                        players[player_index(player + 1)] + 1
                    players[player] = players[player] - 1
                } else {
                    players[player] = players[player] - 1
                }
            }
        }
        return(players)
    }
    
    while ( length( players[ players > 0] ) > 1 ) {
        players <- turn(current_player)
        current_player <- player_index(current_player + 1)
        turns <- turns + 1
    }

    return(turns)
}

all_results <- c()
for (i in 1:simulations) {
    all_results <- c( all_results, game() )
}

M[player_count-5, bills-2] <- mean(all_results)
V[player_count-5, bills-2] <- sd(all_results)
    }
    }
  \end{lstlisting}
\subsection*{Algorithm}

\subsection*{Scripts}

% Commands are                    

% \begin{verbatim*}
% T = diag( 1 ./ [30:-1:1] ) * ( triu( ones(30,30) ) - eye(30) );
% wait = inverse( eye(30) - T) * ones(30,1);
% wait(1)
% HN = sum( 1 ./ [30:-1:1] )
% \end{verbatim*}

% \input{ _scripts}

\hr

\visual{Problems to Work}{../../../../CommonInformation/Lessons/solveproblems.png}
\section*{Problems to Work for Understanding}

\renewcommand{\theexerciseseries}{}
\renewcommand{\theexercise}{\arabic{exercise}}

\begin{exercise}
  Use mathematical induction to show
\[
    P^{m} =
    \begin{pmatrix}
        I_a & 0 \\
        A & T
    \end{pmatrix}
    ^{m-1}
    \begin{pmatrix}
        I_a & 0 \\
        A & T
    \end{pmatrix}
    =
    \begin{pmatrix}
        I_a & 0 \\
        (I_t + T + T^2 + \cdots + T^{m-1})A & T^m
    \end{pmatrix}.
\]
\end{exercise}

\begin{solution}
  
\end{solution}

\begin{exercise}
Show
that the
Euclidean norm of \( T \) is less than \( 1 \), \( \| T \| < 1 \).
\end{exercise}
\begin{solution}
  
\end{solution}
\begin{exercise}
   Show that \( NT =
   (I_t - T)^{-1} T = N - I \)
 \end{exercise}
 \begin{solution}
   
 \end{solution}
\begin{exercise}
  Under the assumption that the
Euclidean norm of \( T \) is less than \( 1 \), \( \| T \| < 1 \)
prove that
\[
  \sum\limits_{x=0}^{\infty} (T^x){}_{ij} = (I - T)^{-1}.
\end{exercise}
\begin{solution}
  
\end{solution}
\hr

\visual{Books}{../../../../CommonInformation/Lessons/books.png}
\section*{Reading Suggestion:}

\bibliography{../../../../CommonInformation/bibliography}

%   \begin{enumerate}
%     \item
%     \item
%     \item
%   \end{enumerate}

\hr

\visual{Links}{../../../../CommonInformation/Lessons/chainlink.png}
\section*{Outside Readings and Links:}
\begin{enumerate}
    \item
        http://www.laurentlessard.com/bookproofs/colorful-balls-puzzle/
    \item
    \item
    \item
\end{enumerate}

\hr

\mydisclaim \myfooter

Last modified:  \flastmod

\end{document}

%%% Local Variables:
%%% mode: latex
%%% TeX-master: t
%%% End:
