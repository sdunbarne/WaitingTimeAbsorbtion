
\begin{description}

% \item[Geogebra] 

% \link{  .ggb}{GeoGebra applet}

% \item[R] 

\link{http://www.math.unl.edu/~sdunbar1/waitingtimeabsorbtion.R}{R
  script for Waiting Time in Left-Center-Right.}

\begin{lstlisting}[language=R]
player_index <- function(j) {
    ((j-1) %% player_count) + 1
}


game <- function(bills, player_count){

    turn <- function(player) {
        rolls <- sample(6, 3, replace=TRUE)
        for (roll in rolls) {
            if (players[player] > 0) {      #player has money
                if (roll <= 2) {            #pass dollar to Left
                    players[player_index(player - 1)] =
                        players[player_index(player - 1)] + 1 
                    players[player] = players[player] - 1
                } else if (roll <= 4) {     #pass dollar to right
                    players[player_index(player + 1)] =
                        players[player_index(player + 1)] + 1
                    players[player] = players[player] - 1
                } else {                    #dollar goes to center
                    players[player] = players[player] - 1
                }
            }
        }
        return(players)
    }

    players <- rep(bills, player_count)
    turns <- 1
    current_player <- 1

    while ( length( players[ players > 0] ) > 1 ) {
        players <- turn(current_player)
        current_player <- player_index(current_player + 1)
        turns <- turns + 1
    }

    return(turns)
}

M <- matrix(0, 4, 4)
V <- matrix(0, 4, 4)
simulations <- 5000

for (player_count in c(6:9)) {
    for (bills in c(3:6)) {

        sim_results <- c()
        for (i in 1:simulations) {
            sim_results <- c( sim_results, game(bills, player_count) )
        }

        M[player_count-5, bills-2] <- mean(sim_results)
        V[player_count-5, bills-2] <- sd(sim_results)
    }
}

\end{lstlisting}  

% \item[Octave]

% \link{http://www.math.unl.edu/~sdunbar1/    .m}{Octave script for .}

% \begin{lstlisting}[language=Octave]

% \end{lstlisting}

% \item[Perl] 

% \link{http://www.math.unl.edu/~sdunbar1/    .pl}{Perl PDL script for .}

% \begin{lstlisting}[language=Perl]

% \end{lstlisting}

% \item[SciPy] 

% \link{http://www.math.unl.edu/~sdunbar1/    .py}{Scientific Python script for .}

% \begin{lstlisting}[language=Python]

% \end{lstlisting}

\end{description}



%%% Local Variables:
%%% mode: latex
%%% TeX-master: t
%%% End:
